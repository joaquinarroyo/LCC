\documentclass{beamer}
\usepackage[spanish]{babel}
\usepackage{graphicx}
\usepackage{xcolor}
\usepackage{latexsym}
\usepackage{amsmath}
\usepackage{amssymb}
\usepackage{multicol}
\usepackage{tikz}

\usetheme{Boadilla}

\newtheorem{Teorema}{Teorema}

\definecolor{myblue}{RGB}{80, 69, 190} 

\setlength{\parskip}{0.1cm}

\newcommand{\indc}[4]{
    \begin{frame}
        \frametitle{Índice - Modelos NoSQL}
        	\begin{enumerate}
            \setbeamercovered{transparent}
            \item<#1> Clave-Valor  
            \item<#2> Columnar      
            \item<#3> Documental    
            \item<#4> Grafo
        \end{enumerate}
    \end{frame}
}

\newcommand{\estindc}[2]{
    \begin{frame}
        \frametitle{Índice - Articulos}
        	\begin{enumerate}
            \setbeamercovered{transparent}
            \item<#1> Assessment of SQL and NoSQL Systems to Store and Mine COVID-19 Data. 
            \item<#2> Performance investigation of selected SQL and NoSQL databases.
        \end{enumerate}
    \end{frame}
}

\newcommand{\myref}[2]{
    {\color{blue}\href{#1}{#2}}
}

\newcommand{\mytitle}[4]{
    \centering
    \vspace{0.1cm}
    
    {\Large \color{myblue} #1}

    \vspace{0.1cm}

    {\Large \color{myblue} #2}
    
    \vspace{0.5cm}
    
    {#3}
    
    \vspace{0.5cm}
    
    {\color{gray} #4}
    
    \vspace{0.5cm}
    
    {\today}
}

\title{Bases de datos NoSQL}
\author{Arroyo Joaquin \\ Belmonte Marina}

\begin{document}

\begin{frame}
    \mytitle
    {Bases de datos NoSQL}
    {Clase 2}
    {Arroyo Joaquin \\ Belmonte Marina}
    {Universidad Nacional de Rosario \\ Licenciatura en Ciencias de la Computación \\ Bases de Datos Avanzadas}
\end{frame}

\section{Resumen}
\begin{frame}
    \frametitle{Resúmen}

    \begin{itemize}
        \item Repaso de Modelos NoSQL

         
        
        \item Implementaciones de Modelos NoSQL

        \begin{itemize}
            \item Redis
            \item Apache HBase
            \item MongoDB
            \item Neo4j
        \end{itemize}

         
        
        \item Dos artículos que estudian el rendimiento de bases de datos SQL versus NoSQL en aplicaciones reales
    \end{itemize}
    
\end{frame}

\section{Repaso}
\begin{frame}
    \frametitle{Repaso}
    Como vimos en la clase 1, existe una amplia variedad de modelos de datos \textit{NoSQL}. Entre los más conocidos encontramos: 
    
     
    
    \begin{enumerate}
        \item Clave-Valor    
        \item Columnar        
        \item Documental     
        \item Grafo
    \end{enumerate}
\end{frame}

\section{Implementaciones}

\indc{1}{0}{0}{0}

\begin{frame}
    \frametitle{Clave-Valor}

    Este modelo almacena pares del tipo (\textit{Key}, \textit{Value}).

    \begin{center}
    \includegraphics[width=0.2\textwidth]{diagramas/KeyValue.png}
    \end{center}
                    
     
        
     \begin{itemize}
        \item \textbf{Key}: Un identificador único para acceder al valor asociado.
         \item \textbf{Value}: Puede ser cualquier tipo de dato, desde texto, números y documentos, hasta listas o incluso otros pares clave-valor.
     
     \end{itemize}

\end{frame}

\begin{frame}
    \frametitle{Clave-Valor - Ventajas}
     \begin{itemize}
     
        \item No hay una estructura de tabla rígida, que permite esquemas flexibles.       
        
        \item Los campos pueden añadirse y modificarse dinámicamente sin alterar el esquema de la base de datos.
        
        \item Permiten el funcionamiento en clústeres, facilitando la adición de nodos para manejar grandes volúmenes de datos.
        
        \item Los motores de procesamiento en paralelo y la arquitectura de clústeres reducen los tiempos de consulta y mejoran la velocidad de escritura y lectura.
     
     \end{itemize}
\end{frame}


\begin{frame}
    \frametitle{Clave-Valor - Ejemplos}
    
    Algunas de las implementaciones más conocidas son:
    
     \begin{itemize}
        \item \textbf{Amazon DynamoDB}\\
        Una base de datos clave-valor y documental de Amazon Web Services. Ofrece alta escalabilidad y disponibilidad.
                        
         
        
        \item \textbf{Voldemort}\\
        Una base de datos distribuida impulsada por LinkedIn, diseñada para ofrecer alta disponibilidad y escalabilidad.
                        
         
        
        \item \textbf{Redis}\\
        Un proyecto de código abierto muy utilizado, especialmente para aplicaciones web modernas. Soporta estructuras de datos avanzadas como listas, sets y hashes.
                        
         
        
        \item \textbf{Riak}\\
        Un sistema escalable que facilita el desarrollo ágil de aplicaciones, con un enfoque en la simplicidad de uso.
     
     \end{itemize}
\end{frame}



% \section{Redis}
% \begin{frame}   
%     \frametitle{Clave-Valor - Redis}

%     \textbf{Redis} que significa ``\textbf{RE}mote \textbf{DI}ctionary \textbf{S}erver'', es una base de datos en memoria de código abierto.

%      

%     Si bien en la teoría, los valores son objetos oscuros, Redis ofrece estructuras de datos para los tipos tales como:

%      

%     \begin{enumerate}
%         \item \textit{Strings} (Valor simple)
%         \item \textit{Hashes}
%         \item \textit{Listas}
%         \item \textit{Conjuntos}
%         \item \textit{Conjuntos Ordenados}  
%         \item etc.
%     \end{enumerate}

%     Se pueden realizar operaciones \textbf{atómicas} sobre estas estructuras.
% \end{frame}

% \begin{frame}
%     \frametitle{Clave-Valor - Redis}
    
%     Redis ofrece más de \textbf{400} operaciones e implementa la interfaz teórica para cada uno de los tipos de datos mencionados (además de ofrecer otras operaciones).

%      

%     Algunos ejemplos son:

%      

%     \begin{itemize}
%         \item \textit{Strings:} $SET$, $GET$ y $DEL$

%          
        
%         \item \textit{Hashes:} $HSET$, $HGET$ y $HDEL$

%          
        
%         \item \textit{Listas:} $LPUSH$, $LPOP$, $RPUSH$, $RPOP$, $LRANGE$
%          

%         \item etc.
%     \end{itemize}

%      
    
%     La principal diferencia entre estas operaciones, es su complejidad temporal,   y hay que tenerla en cuenta.

%      

%     Hay que tener en claro el nivel de representatividad que se necesita, y capacidad de memoria que se posee, para elegir correctamente la estructura a utilizar.
% \end{frame}

% \begin{frame}[fragile]
%     \frametitle{Clave-Valor - Redis}
    
%     Algunos ejemplos de las operaciones mencionadas:

%      

%     \begin{verbatim}
%     redis> LPUSH mylist "NoSQL"
%     redis> LPUSH mylist "Datos" "De" "Bases"
%     redis> LRANGE mylist 0 2
%     1) "Bases"
%     2) "De"
%     3) "Datos"
%     redis>
%     \end{verbatim}

%      

%     \vspace{-0.8cm}
    
%     \begin{verbatim}
%     redis> ZADD myzset 2 "b" 3 "c"
%     (integer) 2
%     redis> ZADD myzset 4 "a"
%     redis> ZMEMBERS myset BYLEX
%     1) "a"
%     2) "b"
%     3) "c"
%     \end{verbatim}
    
% \end{frame}

\indc{0}{1}{0}{0}

\subsection{Columnar}

\begin{frame}
    \frametitle{Columnar - Repaso}
    
    \begin{columns}
        \begin{column}{0.64\textwidth}
            Las bases de datos orientadas a columnas almacenan datos verticalmente por columnas en lugar de horizontalmente por filas, permitiendo un acceso más eficiente a datos específicos.
        \end{column}
        \begin{column}{0.34\textwidth}
            \centering
            \includegraphics[width=0.7\textwidth]{images/Columnar.png}
        \end{column}
    \end{columns}

     
    
    Algunas de sus ventajas:

     
    
    \begin{itemize}
        \item  Mayor eficiencia en consultas sobre columnas específicas.  
        \item  Convenientes para análisis de grandes volúmenes de datos.  
        \item  Adecuadas para aplicaciones de business intelligence y análisis predictivo.
    \end{itemize}
    
\end{frame}

\begin{frame}
    \frametitle{Columnar - Ejemplos}
    Algunos ejemplos de bases de datos columnares son: 
    
    \begin{itemize}
        \item \textbf{Apache Cassandra}
        
        \item \textbf{Apache HBase}
        
        \item \textbf{ClickHouse}
    \end{itemize}

     
    
    Nos vamos a centrar en Apache HBase.
\end{frame}

\subsubsection{Apache HBase}

\begin{frame}
    \frametitle{Columnar - Apache HBase}
    \centering
    \includegraphics[width=0.4\textwidth]{images/hbase-logo.png}
    \begin{itemize}
        \item Base de datos distribuida y escalable.  
        \item Desarrollada por la Apache Software Foundation.  
        \item Modelo de datos basado en tablas y columnas flexibles.  
        \item Almacenamiento de datos en archivos HFile en Hadoop HDFS.  
        \item Utiliza partición distribuida y replicación para alta disponibilidad.  
        \item Ofrece diferentes niveles de consistencia.
    \end{itemize}

\end{frame}


\begin{frame}{Operaciones Básicas en Hbase}

HBase proporciona un conjunto de comandos específicos que se utilizan para realizar diversas operaciones administrativas y de consulta.
      

    Algunos ejemplos son:

      
    \begin{enumerate}
        \item \textbf{create}: Crear una nueva tabla.  
        \item \textbf{put}: Insertar un valor en una celda de datos.  
        \item \textbf{get}: Recuperar datos de una fila.  
        \item \textbf{delete}: Eliminar una celda de datos.  
        \item \textbf{drop}: Eliminar una tabla.
    \end{enumerate}

\end{frame}

\begin{frame}[fragile]
\frametitle{Ejemplo de Uso}
\begin{verbatim}
hbase(main):001:0> create 'usuarios', 
                    'datos_personales', 'datos_contacto'
0 fila(s) en 1.2340 segundos
\end{verbatim}

\end{frame}

\begin{frame}[fragile]
\frametitle{Ejemplo de Uso}
\begin{verbatim}

hbase(main):002:0> put 'usuarios', '1001', 
'datos_personales:nombre', 'Juan'
0 fila(s) en 0.0510 segundos

hbase(main):003:0> put 'usuarios', '1001', 
'datos_personales:apellido', 'Pérez'
0 fila(s) en 0.0110 segundos

hbase(main):004:0> put 'usuarios', '1001', 
'datos_contacto:email', 'juan@example.com'
0 fila(s) en 0.0090 segundos
\end{verbatim}

\end{frame}

\begin{frame}[fragile]
\frametitle{Ejemplo de Uso}
\begin{verbatim}
hbase(main):005:0> get 'usuarios', '1001', 
{COLUMN => 'datos_personales'}
COLUMN                          CELL
 datos_personales:nombre        valor=Juan
 datos_personales:apellido      valor=Pérez
1 fila(s) en 0.0190 segundos
\end{verbatim}
\end{frame}

\begin{frame}[fragile]
\frametitle{Ejemplo de Uso}
\begin{verbatim}

hbase(main):006:0> delete 'usuarios', '1001', 
'datos_contacto:email'
0 fila(s) en 0.0230 segundos
\end{verbatim}
  
\begin{verbatim}
hbase(main):007:0> get 'usuarios', '1001'
COLUMN                          CELL
 datos_personales:nombre        valor=Juan
 datos_personales:apellido      valor=Pérez
1 fila(s) en 0.0190 segundos
\end{verbatim}

\end{frame}

\begin{frame}{Operaciones Avanzadas en HBase}
    \begin{enumerate}
        \item Escaneo de Rango  
        \item Filtros de Columnas y Filas  
        \item Transacciones y Consistencia  
        \item Replicación de Datos  
        \item Optimización de Rendimiento  
        \item Seguridad
    \end{enumerate}
\end{frame}

\indc{0}{0}{1}{0}

\begin{frame}
    \frametitle{Documental}

    Las bases de datos documentales almacenan datos en documentos individuales, que pueden ser estructurados o semi-estructurados como archivos JSON o XML.
    
    \begin{center}
    \includegraphics[width=0.2\textwidth]{diagramas/Documental.png}
    \end{center}
\end{frame}

\begin{frame}
    \frametitle{Documental - Ventajas}

    \begin{itemize}
        \item Las bases de datos son flexibles.  
        \item Los sistemas pueden crear índices sobre algunos elementos de datos para mejorar el rendimiento de las consultas.
    \end{itemize}
\end{frame}

\begin{frame}
    \frametitle{Documental - Casos de uso}
    Las bases de datos documentales son especialmente adecuadas para aplicaciones donde los datos tienen una estructura flexible y variable como por ejemplo:
    
     
    
    \begin{itemize}
        \item Contenido web  
        \item Análisis de registros  
        \item Gestión de datos de productos
    \end{itemize}
\end{frame}

\begin{frame}
    \frametitle{Documental - Ejemplos}
    \begin{itemize}
        \item \textbf{MongoDB}\\
        Es una base de datos documental líder que utiliza un modelo de datos basado en documentos JSON. Ofrece capacidad de escalabilidad horizontal y opciones avanzadas de replicación, lo que permite gestionar grandes cantidades de datos y cargas de trabajo exigentes.

         

        \item \textbf{CouchDB}\\
        Se destaca por su simplicidad en la replicación y consulta. Utiliza un modelo de datos basado en documentos JSON y es especialmente adecuado para cargas de trabajo menos intensivas y entornos donde la replicación descentralizada es una necesidad.
    \end{itemize}
\end{frame}

\indc{0}{0}{0}{1}

\begin{frame}
    \frametitle{Grafo}

    Las bases de datos en grafo utilizan estructuras de grafo para almacenar, consultar y relacionar datos. 

    \begin{center}
    \includegraphics[width=0.2\textwidth]{diagramas/Grafo.png}
    \end{center}
    
     
    
    Los datos se representan mediante nodos (entidades) y arcos (relaciones entre entidades).  
    
    Pueden asignarse propiedades a nodos y arcos para capturar más detalles.
\end{frame}

\begin{frame}
    \frametitle{Grafo - Casos de uso}

    Las bases de datos en grafo son ideales para almacenar datos interconectados.  
    Aplicaciones comunes incluyen:
    \begin{itemize}
        \item Redes sociales\\
          
        \item Sistemas de recomendación\\
         
        \item Redes de trasportes\\
        
    \end{itemize}

    Los datos almacenados pueden interpretarse de diferentes maneras basadas en las relaciones entre los nodos.
\end{frame}

\begin{frame}
    \frametitle{Grafo - Desventajas}
    \begin{itemize}
        \item Las bases de datos en grafo presentan complejidad de modelado y un costo de aprendizaje elevado para usuarios no familiarizados.

         

        \item Pueden experimentar dificultades de rendimiento y requerir esfuerzos adicionales de mantenimiento y optimización.
    \end{itemize}
\end{frame}


\begin{frame}
    \frametitle{Grafo - Ejemplos}
    Algunos ejemplos de bases de datos en grafos son:
    
     

    \begin{itemize}
        \item \textbf{Neo4J}\\
        Una de las bases de datos en grafo más populares y utilizadas en la industria. Ofrece una gran variedad de herramientas y bibliotecas para el modelado y análisis de grafos.

         

        \item \textbf{GraphBase}\\ 
        Un sistema que permite crear y consultar grafos de datos, con capacidades avanzadas para el análisis de datos.

         
        
        \item \textbf{Infinite Graph}\\
        Ofrece una plataforma para trabajar con grafos a gran escala, con soporte para consultas complejas y análisis de grafos.    
        
         

        \item \textbf{FlockDB}\\
        Un sistema de bases de datos en grafo diseñado para manejar relaciones entre grandes volúmenes de datos, con un enfoque en el rendimiento y la escalabilidad.
    \end{itemize}
    
\end{frame}

\section{Estudios}

\estindc{1}{0}

\subsection{Assessment of SQL and NoSQL Systems to Store and
Mine COVID-19 Data}

\begin{frame}
    \frametitle{Assessment of SQL and NoSQL Systems to Store and Mine COVID-19 Data}

    Este estudio fue llevado a cabo en el año 2022 por tres investigadores pertenecientes a los siguientes institutos

    \vspace{-0.3cm}
    
    \begin{center}
        \textit{Instituto de Ingeniería de Coimbra}$^1$, \textit{Centro de Informática y Sistemas de la Universidad de Coimbra}$^2$ e \textit{Instituto Tecnológico de São Paulo}$^3$
    \end{center}

    \vspace{-0.4cm}
    
    \begin{center}
        João Antas$^1$, Rodrigo Rocha Silva$^{2,3}$ y Jorge Bernardino$^{1,2}$
    \end{center}

    \vspace{-0.1cm}
    
    y tuvo como objetivo:

     

    \begin{itemize}
        \item Estudiar diferentes sistemas de bases de datos, con el fin de ayudar a seleccionar el más adecuado para almacenar, gestionar y minar datos relacionados con el COVID-19.

         
        
        \item Llevar a cabo un proceso de Minería de Datos, empleando pruebas de clasificación de datos del software Orange Data Mining.
    \end{itemize}

     

    Nos vamos a centrar en mostrar como se logró el primer objetivo.
\end{frame}

\begin{frame}
    \frametitle{Assessment of SQL and NoSQL Systems to Store and Mine COVID-19 Data}

    Para ambos experimentos, utilizaron una computadora con las siguiente características:

     
    
    \begin{itemize}
        \item Windows 10
        \item Procesador Intel Core i7-8750H 2.20GHz
        \item 16 GB de RAM
        \item 256 GB SSD de Almacenamiento
    \end{itemize}

     
    
    Además las versiones de las tecnologías utilizadas fueron las siguientes:

     
    
    \begin{itemize}
        \item SQL Server versión 2017
        \item MongoDB versión 4.4
        \item Cassandra versión 3.11.10
    \end{itemize}
\end{frame}

\begin{frame}
    \frametitle{Assessment of SQL and NoSQL Systems to Store and Mine COVID-19 Data}

    Los datos fueron obtenidos de distintas fuentes, como hospitales, datos públicos, etc., y fueron almacenados en formatos CSV o XML antes de ser incorporados a las bases de datos.   Y para evaluar la escalabilidad de las bases de datos, crearon dos datasets de distintos tamaños. 

     
    
    Para el primer experimento, utilizaron seis queries diferentes, con el objetivo de evaluar el tiempo de ejecución, la RAM utilizada y el porcentaje de CPU de cada consulta. 
    
     
    
    Vamos a mostrar los resultados obtenidos sobre las consultas 1, 2 y 3, que se presentan a continuación.

\end{frame}

\begin{frame}
    \frametitle{Assessment of SQL and NoSQL Systems to Store and Mine COVID-19 Data}

    \begin{figure}[H]
        \begin{center}
            \includegraphics[width=0.7\textwidth]{images/cov19-query12.png}
            \caption{Query Region (Query 1 - sin subrayado) y Query RegionYear (Query 2)}
            \label{cov19-q12}
        \end{center}
    \end{figure}
\end{frame}

\begin{frame}
    \frametitle{Assessment of SQL and NoSQL Systems to Store and Mine COVID-19 Data}

    Esta query fue seleccionada utilizando un registro de auditoría que controlaba todas las consultas realizadas en la base de datos de SQL Server.
    
    \begin{figure}[H]
        \begin{center}
            \includegraphics[width=0.82\textwidth]{images/cov19-query3.png}
            \caption{Query de Orange Data Mining (Query 3)}
            \label{cov19-q3}
        \end{center}
    \end{figure}
\end{frame}

\begin{frame}
    \frametitle{Assessment of SQL and NoSQL Systems to Store and Mine COVID-19 Data}

    \begin{figure}[H]
        \centering
        \begin{minipage}[b]{0.48\textwidth}
            \centering
            \includegraphics[width=\textwidth]{images/cov19-runt-test-q1.png}
            \caption{Prueba de tiempo de ejecución para la Query 1.}
        \end{minipage}
        \hfill
        \begin{minipage}[b]{0.48\textwidth}
            \centering
            \includegraphics[width=\textwidth]{images/cov19-runt-test-q2.png}
            \caption{Prueba de tiempo de ejecución para la Query 2.}
        \end{minipage}
    \end{figure}
\end{frame}

\begin{frame}
    \frametitle{Assessment of SQL and NoSQL Systems to Store and Mine COVID-19 Data}

    \begin{figure}[H]
        \centering
        \begin{minipage}[b]{0.48\textwidth}
            \centering
            \includegraphics[width=\textwidth]{images/cov19-mem-test-q1.png}
            \caption{Memoria RAM utilizada para la Query 1.}
            \label{cov19-memtest-q1}
        \end{minipage}
        \hfill
        \begin{minipage}[b]{0.48\textwidth}
            \centering
            \includegraphics[width=\textwidth]{images/cov19-mem-test-q2.png}
            \caption{Memoria RAM utilizada para la Query 2.}
            \label{cov19-memtest-q2}
        \end{minipage}
    \end{figure}
\end{frame}

\begin{frame}
    \frametitle{Assessment of SQL and NoSQL Systems to Store and Mine COVID-19 Data}

    \begin{figure}[H]
        \centering
        \begin{minipage}[b]{0.48\textwidth}
            \centering
            \includegraphics[width=\textwidth]{images/cov19-cpu-test-q1.png}
            \caption{Porcentaje de CPU utilizado para la Query 1.}
            \label{cov19-cputest-q1}
        \end{minipage}
        \hfill
        \begin{minipage}[b]{0.48\textwidth}
            \centering
            \includegraphics[width=\textwidth]{images/cov19-cpu-test-q2.png}
            \caption{Porcentaje de CPU utilizado para la Query 2.}
            \label{cov19-cputest-q2}
        \end{minipage}
    \end{figure}
\end{frame}

\begin{frame}
    \frametitle{Assessment of SQL and NoSQL Systems to Store and Mine COVID-19 Data}

    \begin{figure}[H]
        \centering
        \begin{minipage}[b]{0.48\textwidth}
            \centering
            \includegraphics[width=\textwidth]{images/cov19-runt-test-q3.png}
            \caption{Prueba de tiempo de ejecución para la Query 3.}
            \label{cov19-runtest-q3}
        \end{minipage}
        \hfill
        \begin{minipage}[b]{0.48\textwidth}
            \centering
            \includegraphics[width=\textwidth]{images/cov19-mem-test-q3.png}
            \caption{Memoria RAM utilizada para la Query 3.}
            \label{cov19-memtest-q3}
        \end{minipage}
    \end{figure}
\end{frame}

\begin{frame}
    \frametitle{Assessment of SQL and NoSQL Systems to Store and Mine COVID-19 Data}

    \begin{figure}[H]
        \centering
        \begin{minipage}[b]{0.48\textwidth}
            \centering
            \includegraphics[width=\textwidth]{images/cov19-cpu-test-q3.png}
            \caption{Porcentaje de CPU utilizado para la Query 3.}
            \label{cov19-runtest-q3}
        \end{minipage}
        \hfill
        \begin{minipage}[b]{0.48\textwidth}
            \centering
            \includegraphics[width=\textwidth]{images/cov19-storage.png}
            \caption{Tamaño de la base de datos de COVID-19.}
            \label{cov19-memtest-q3}
        \end{minipage}
    \end{figure}
\end{frame}

\subsubsection{Conclusiones}

\begin{frame}
    \frametitle{Assessment of SQL and NoSQL Systems to Store and Mine COVID-19 Data - Conclusiones}

    Cabe destacar que los experimentos que no se mostraron, utilizaron consultas con \textit{joins}, y SQL Server tuvo un mejor rendimiento.

     

    \begin{itemize}
        \item SQL Server debería ser la elección si los datos son muy estructurados y necesitan consultas con \textit{joins}.

         
    
        \item Si se utiliza un gran volumen de datos no estructurados y no es necesario realizar demasiadas consultas con \textit{joins}, MongoDB o Cassandra se consideran las soluciones más adecuadas.
    \end{itemize}
\end{frame}

\estindc{0}{1}

\subsection{Performance investigation of selected SQL and NoSQL databases}

\begin{frame}
    \frametitle{Performance investigation of selected SQL and NoSQL databases}

    Este artículo fue presentado en el año 2015 en la Conferencia Internacional sobre Ciencia de la Información Geográfica (AGILE) por tres investigadores de la \textit{Universidad de Bundeswehr}:

     
    
    \begin{center}
        Stephan Schmid, Eszter Galicz y Wolfgang Reinhardt.
    \end{center}
    
     

    \begin{itemize}
        \item Trata sobre la creciente importancia de los datos espaciales, en el mundo actual.

         

        \item Explora las bases de datos NoSQL como una posible alternativa ante el dominio de las bases de datos relacionales para almacenar y manipular este tipo de datos.
    \end{itemize}
    
     
    
    Para los experimentos utilizaron tres bases de datos:

    \begin{center}
        \textbf{PostgreSQL}, \textbf{MongoDB} y \textbf{CouchBase}.
    \end{center}
\end{frame}

\begin{frame}
    \frametitle{Performance investigation of selected SQL and NoSQL databases}

    Para la representación de datos espaciales en PostgreSQL utilizaron \textbf{PostGis},   y en las dos siguientes utilizaron el formato \textbf{GeoJSON}, el cuál permite representar \textbf{Geometrías} (Puntos, Polígonos, Colección de Geometrías, etc.), \textbf{Características} y \textbf{Coleciones de Características}.

     

    Mencionan que al utilizar las estructuras de datos GeoJSON, el enfoque sin esquemas tiene algunas restricciones.
    
     
    
    Sin embargo, la representación geográfica debe seguir dicha estructura para poder establecer un índice geoespacial.
    
\end{frame}

\begin{frame}
    \frametitle{Performance investigation of selected SQL and NoSQL databases}

    Para los experimentos, los autores utilizaron una computadora con las siguientes características:

    \begin{itemize}
        \item Microsoft Windows Server 2008 R2
        \item 8 core CPU 2,5 GHz
        \item 10GB RAM
    \end{itemize}

     
    
    Utilizaron tres datasets para los experimentos, los cuales fueron obtenidos OpenStreetMap.

     
    
    \begin{table}[h]
        \centering
        \begin{tabular}{ |c|c|c| }
        \hline
        Level & Region & Size \\ 
        \hline
        Subregion & Niederbayern & 38.9 MB \\
        State & Bayern & 501 MB \\ 
        Country & Germany & 2.1 GB \\
        \hline
        \end{tabular}
        \caption{Datos de prueba utilizados de OpenStreetMap.}
    \end{table}
    
\end{frame}

\begin{frame}
    \frametitle{Performance investigation of selected SQL and NoSQL databases}
    
    \begin{minipage}{0.55\textwidth}

        Y eligieron dos tipos de consultas para el análisis:

        \begin{enumerate}
            \item Consultas sobre información de atributos.
    
            \item Consultas que utilizan la geo-función \textit{within}.
        \end{enumerate}
        
        \begin{figure}
            \centering
            \includegraphics[width=\textwidth]{images/geo_q1.png}
            \caption{Query 1}
        \end{figure}
    \end{minipage}\hfill
    \begin{minipage}{0.43\textwidth}
        \begin{figure}
            \centering
            \includegraphics[width=\textwidth]{images/geo_q2.png}
            \caption{Query 2}
        \end{figure}
    \end{minipage}
    
\end{frame}

\begin{frame}
    \frametitle{Performance investigation of selected SQL and NoSQL databases}

    Para realizar la simulación en condiciones realistas, las consultas se realizaron con una determinada cantidad de usuarios, la cuál va en aumento:

    \vspace{-0.4cm}
    
    \begin{center}
        100, 250 y 500 usuarios.
    \end{center}

     

    \begin{minipage}{0.48\textwidth}
        \begin{figure}
            \centering
            \includegraphics[width=\textwidth]{images/geo-g1.png}
            \caption{Resultados Query 1}
        \end{figure}
    \end{minipage}\hfill
    \begin{minipage}{0.48\textwidth}
        \begin{figure}
            \centering
            \includegraphics[width=\textwidth]{images/geo-g2.png}
            \caption{Resultados Query 2}
        \end{figure}
    \end{minipage}
\end{frame}

\subsubsection{Conclusiones}

\begin{frame}
    \frametitle{Performance investigation of selected SQL and NoSQL databases - Conclusión}

    \begin{itemize}
        \item Las consultas con el uso de geo-funciones llevan más tiempo que las consultas sobre información de atributos.

         
        
        \item Para solicitudes puramente sobre información de atributos, las bases de datos NoSQL son superiores en comparación con las bases de datos SQL.

         
        
        \item Los resultados muestran claramente que las bases de datos NoSQL son una alternativa posible, al menos para consultar información de atributos.
    \end{itemize}
\end{frame}

\section{Dudas}

\begin{frame}
    \vspace{1cm}
    
    \centering
    Dudas?
\end{frame}

\section{Referencias}
\begin{frame}
\frametitle{Referencias}
    \begin{itemize}
    	\item[1] \myref{https://redis.io/es/}{redis.io} Accedido el 15.04.2024

        \item[2] \myref{https://hbase.apache.org/}{hbase.apache.org} Accedido el 13.05.2024

        \item[3] \myref{https://www.mongodb.com/es}{mongodb.com} Accedido el 13.05.2024
        
        \item[4] \myref{https://neo4j.com/}{neo4j.com} Accedido el 06.05.2024
        
        \item[5] Antas, J., Silva, R.R., Bernardino, J.: Assessment of sql and nosql systems to store and mine covid-19 data. MDP Comput. Suv. (2022)

        \item[6] Schmid, S., Galicz, E., Reinhardt, W.: Performance investigation of selected sql and nosql databases (2015)

        \item[7] \myref{https://datatracker.ietf.org/doc/html/rfc7946}{datatracker.ietf.org/doc} Accedido el 10.05.2024
        
    \end{itemize}
\end{frame}

\end{document}