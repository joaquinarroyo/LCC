\subsection{Documental}

\begin{frame}
    \frametitle{Documental}

    \begin{columns}
        \begin{column}{0.48\textwidth}
            Las bases de datos documentales almacenan datos en documentos individuales, que pueden ser estructurados o semi-estructurados como archivos JSON o XML.
        \end{column}
        \begin{column}{0.48\textwidth}
            \centering
            \includegraphics[width=0.5\textwidth]{images/Documental.png}
        \end{column}
    \end{columns}
    
     

    \vspace{0.3cm}
    
    Son especialmente adecuadas para aplicaciones donde los datos tienen una estructura flexible y variable como por ejemplo:

     
    
    \begin{itemize}
        \item \textbf{Contenido web}  
        \item \textbf{Análisis de registros}  
        \item \textbf{Gestión de datos de productos}
        \item etc.
    \end{itemize}
\end{frame}

\begin{frame}
    \frametitle{Documental - Ejemplos}

    Algunos ejemplos de bases de datos documentales son:
    
    \begin{itemize}
        \item \textbf{MongoDB}

        \item \textbf{CouchDB}
    \end{itemize}

     
    
    Nos vamos a centrar en MongoDB.
\end{frame}

\subsubsection{MongoDB}

\begin{frame}
    \frametitle{Documental - MongoDB}

        \centering
    \includegraphics[width=0.4\textwidth]{images/mongodb-logo.png}
    \begin{itemize}
        \item Base de datos de código abierto, orientada a documentos y altamente escalable.  
        \item Desarrollada por MongoDB Inc.  
        \item Modelo de datos flexible basado en documentos BSON.  
        \item Utiliza almacenamiento basado en archivos de mapeo directo o motor WiredTiger.  
        \item Maneja replicación a través de conjuntos de réplicas para alta disponibilidad.  
        \item Distribuye datos en clústeres de servidores llamados fragmentos.  
        \item Lenguaje de consulta poderoso y flexible.
    \end{itemize}

\end{frame}

\begin{frame}[fragile]
\frametitle{Operaciones comunes en MongoDB}
\begin{itemize}
    \item \textbf{Insertar un documento}
    \begin{verbatim}
db.users.insertOne({
    name: "John Doe",
    age: 30,
    email: "john@example.com"
})

db.users.insertMany([
  { name: "Alice", age: 25 },
  { name: "Bob", age: 30 },
  { name: "Charlie", age: 50 }
])
    \end{verbatim}

\end{itemize}
\end{frame}

\begin{frame}[fragile]
\frametitle{Operaciones comunes en MongoDB}
\begin{itemize}
    \item \textbf{Consultar documentos}
    \begin{verbatim}
db.users.find({ age: { $gt: 25 } })
    \end{verbatim}

     
    \item \textbf{Actualizar documentos}
    \begin{verbatim}
db.users.updateOne(
    { name: "John Doe" },
    { $set: { age: 35 } }
)

db.users.updateMany(
    { status: "active" }, 
    { $set: { status: "inactive" } }
)

    \end{verbatim}
\end{itemize}
\end{frame}

\begin{frame}[fragile]
\frametitle{Operaciones comunes en MongoDB}
\begin{itemize} 
    \item \textbf{Eliminar documentos}
    \begin{verbatim}
db.users.deleteOne({ name: "John Doe" })
db.users.deleteMany({ age: { $gt: 40 } })
    \end{verbatim}
         
    \item \textbf{Contar documentos}
\begin{verbatim}
db.users.countDocuments({ age: { $lt: 30 } })
\end{verbatim}
         
    \item \textbf{Valores Distintos}
\begin{verbatim}
db.users.distinct("city", { country: "USA" })
\end{verbatim}
         
    \item \textbf{Agregar datos}
\begin{verbatim}
db.sales.aggregate([
  { $match: { status: "completed" } },
  { $group: { _id: "$product", 
            totalAmount: { $sum: "$amount" } } }
])
\end{verbatim}

\end{itemize}

\end{frame}