\begin{frame}
    \frametitle{Fraud Detection in NoSQL Database Systems using Advanced Machine Learning}

    \begin{itemize}
        \item Autor: Tamilselvan Arjunan.
         
        \item Publicado en: International Journal of Innovative Science and Research Technology.
        \item Fecha: Marzo 2024.
         
        \item Análisis de vulnerabilidades en MongoDB y Cassandra.
         
        \item Propuesta de algoritmos de aprendizaje automático.
    \end{itemize}
\end{frame}

\begin{frame}
    \frametitle{Fraud Detection in NoSQL Database Systems using Advanced Machine Learning - Problemas de Seguridad en NoSQL}
    \begin{itemize}
        \item Esquemas Dinámicos.
         
        \item Falta de Control de Acceso.
         
        \item Consistencia Eventual.
         
        \item Datos Desnormalizados.
         
        \item Configuraciones Inseguras por Defecto.
    \end{itemize}
\end{frame}

\begin{frame}
    \frametitle{Fraud Detection in NoSQL Database Systems using Advanced Machine Learning - Soluciones Propuestas}
    \begin{itemize}
        \item Monitoreo en Tiempo Real.
         
        \item Algoritmos de Aprendizaje Automático:
        \begin{itemize}
            \item Modelos Supervisados.
             
            \item Técnicas No Supervisadas.
             
            \item Métodos en Línea.
             
        \end{itemize}
        \item Ingeniería de Características.
         
        \item Sistema Híbrido de Detección.
         
        \item Aprendizaje Adversarial.
         
        \item Implementación y Despliegue.
    \end{itemize}
\end{frame}

\begin{frame}
    \frametitle{Fraud Detection in NoSQL Database Systems using Advanced Machine Learning - Conclusiones}
    \begin{itemize}
        \item Las técnicas avanzadas de aprendizaje automático mejoran la seguridad.
         
        \item Mejorar los mecanismos de control de acceso y autenticación permitirían reducir la superficie de ataque.
         
        \item Superar desafíos de integración y necesidad de datos etiquetados.
         
        \item  Fomentar la colaboración entre equipos de desarrollo y seguridad para asegurar la implementación efectiva de medidas de protección.
    \end{itemize}
\end{frame}
